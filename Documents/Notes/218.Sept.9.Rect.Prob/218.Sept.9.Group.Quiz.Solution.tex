\documentclass[11pt]{article}

\usepackage{latexsym}
\usepackage{amssymb}
\usepackage{amsthm}
\usepackage{enumerate}
\usepackage{amsmath}
\usepackage{cancel}

\setlength{\evensidemargin}{.25in}
\setlength{\oddsidemargin}{-.25in}
\setlength{\topmargin}{-.75in}
\setlength{\textwidth}{6.5in}
\setlength{\textheight}{9.5in}
\newcommand{\due}{September 12th, 2015}
\newcommand{\HWnum}{1}
\input{"C:/Users/joeor_000/Documents/Coursework/Define.tex"}

\begin{document}
A train with an initial velocity $v_0$ starts breaking at position $x_0$. It experiences a 
deceleration of the form 
\begin{equation}
a(t) = -ct^2. 
\label{aEqn}
\end{equation}
For safety reasons we wish to keep the magnitude of deceleration under a maximum value given by $a_{max} = g/3$. With this information we can 
calculate the distance we need to start breaking such that the train comes to a stop without
the magnitude of the deceleration exceeding $a_{max}$. Note that we made $a(t)$ be a 
negative quantity because the breaks slow down the train. This is an arbitrary choice that
we can make either way without changing the final solution.

First we need to determine the initial conditions of the system. An initial condition is the
state of the system at the start of the time or when $t=0$. So we are told that we start 
with an initial velocity of $v_0$ at position $x_0$ therefore we say
\begin{align*}
v(t=0) &= v_0\\
x(t=0) &= x_0.
\end{align*}
Here we are explicitly assuming that velocity, $v(t)$, and position, $x(t)$, are functions of
time, the function form of which we will need to determine. 

First to determine the functional form of velocity, $v(t)$, we use the fact that acceleration
is the change of velocity with respect to time. Mathematically we write this as
\begin{equation}
a(t) = \frac{dv(t)}{dt}
\label{Accler}
\end{equation}
where $d/dt$ is the derivative with respect to time. Therefore we can rearrange equation
\ref{Accler} by multiplying both sides by $dt$
\begin{align*}
a(t) &= \frac{dv(t)}{dt}\\
&\Downarrow\\
a(t)dt &= \frac{dv(t)}{\cancel{dt}}\cancel{dt}\\
&\Downarrow\\
a(t)dt &= dv(t)
\end{align*}
Now to we can solve for $v(t)$ by integrating both sides where we use the given function of 
acceleration shown in equation \ref{aEqn}.
\begin{align*}
\int dv(t) &= \int a(t)dt \\
&\Downarrow\\
v(t) &= \int -ct^2dt \\
&= -\frac{c}{3}t^3 + A
\end{align*}
Note that we gained a new constant, $A$, from integrating. This is called a constant of 
integration. We can solve for $A$ by using the initial condition $v(0) = v_0$. This yields
\begin{align*}
v(t=0) &= v_0 = \cancelto{0}{-\frac{c}{3}(0)^3} + A\\
v_0 &= A
\end{align*}
We note that the integration constant in this case was just our initial velocity. So now we
have a function that describes our velocity as a function of time given by
\begin{equation}
v(t) = -\frac{c}{3}t^3 + v_0.
\label{vEqn}
\end{equation}

Next we need to determine the functional form of position, $x(t)$. This process follows in 
the same was we determined $v(t)$. Namely we use the fact that velocity is the change of 
position with respect to time or mathematically
\begin{equation}
v(t) = \frac{dx(t)}{dt}
\label{Velo}
\end{equation}
Like before, we rearrange equation \ref{Velo} by multiplying both sides by $dt$ to get
\begin{align*}
v(t) &= \frac{dx(t)}{dt}\\
&\Downarrow\\
v(t)dt &= \frac{dx(t)}{\cancel{dt}}\cancel{dt}\\
&\Downarrow\\
v(t)dt &= dx(t)
\end{align*}
Now to solve for $x(t)$ we integrate both sides of this equations using the formula for 
$v(t)$ we found above that is shown in equation \ref{vEqn}. So,
\begin{align*}
\int dx(t) &= \int v(t)dt \\
&\Downarrow\\
x(t) &= \int\left(-\frac{c}{3}t^3 + v_0\right)dt\\
&= \int-\frac{c}{3}t^3dt + \int v_0dt\\
&= -\frac{c}{12}t^4 + v_0t + B
\end{align*}
Again we gained a constant of integration, $B$, and like before we use our initial condition
$x(0) = x_0$ to determine $B$ by
\begin{align*}
x(t=0) &= x_0 = \cancelto{0}{-\frac{c}{12}(0)^4} + \cancelto{0}{v_0(0)} + B\\
x_0 &= B
\end{align*}
So we found that $B=x_0$. Now we have found another function this time that function 
describes our position as a function of time. This is given by
\begin{equation}
x(t) = -\frac{c}{12}t^4 + v_0t + x_0.
\label{xEqn}
\end{equation}

Grouping the equations that describe the motion of our train together we have
\begin{align*}
x(t) &= -\frac{c}{12}t^4 + v_0t + x_0\\
v(t) &= -\frac{c}{3}t^3 + v_0\\
a(t) &= -ct^2
\end{align*}

Now we are ready to start finding the distance we need to start breaking at! To do this we 
need to realize ultimately the train needs to come to a stop. This means at a final time we
will call $t_f$ the velocity of the train needs to be zero. Mathematically we say this by 
$v(t=t_f) = 0$. We realize that $t_f$ is an important quantity to finding the distance we 
need to stop breaking at, because the distance the train has traveled once we have come to a
stop is $x(t=t_f) - x_0$. So we need to determine the final time, $t_f$. We can do this by 
taking the velocity function we found to be equation \ref{vEqn} and evaluating it at $t=t_f$,
and solving for $t_f$.
\begin{align*}
v(t=t_f) = 0 &= -\frac{c}{3}(t_f)^3 + v_0\\
&\Downarrow\\
-v_0 &= -\frac{c}{3}t_f^3 \\
v_0\frac{3}{c} &= \cancel{-\frac{c}{3}}t_f^3\cancel{-\frac{3}{c}}\\
&\Downarrow\\
t_f^3 &= \frac{3v_0}{c}\\
\left(t_f^3\right)^{1/3} &= \left(\frac{3v_0}{c}\right)^{1/3}\\
&\Downarrow\\
t_f &= \left(\frac{3v_0}{c}\right)^{1/3}
\end{align*}
So we have found $t_f$ but we are not done yet! We see that the final time, $t_f$, depends 
on the constant $c$ that was given as part of equation \ref{aEqn}. To find $c$ we need to 
use the safety constraint that states that $a_{max} = g/3$. We reason that if we want to 
ensure that we do not exceed $a_{max}$ we need to reach $a_{max}$ at time final. This means
that $a(t=t_f) = g/3$. So using equation \ref{aEqn}, which is the given functional form of 
acceleration, evaluated at $t=t_f$ we can solve for $c$.
\begin{align*}
a(t=t_f) = -\frac{g}{3} &= -ct^2_f\\
&\Downarrow\\
\frac{g}{3} &= ct^2_f\\
\frac{g}{3}\frac{1}{t_f^2} &= c\cancel{t_f^2}\cancel{\frac{1}{t_f^2}}\\
&\Downarrow\\
c &= \frac{g}{3t_f^2}
\end{align*}
We note that $c$ has a $t_f^2$ in it. This means we are going to need to do some extra work
when we replace $c$ in our first solution of $t_f$.
\begin{align*}
t_f &= \left(\frac{3v_0}{c}\right)^{1/3}\\
&\Downarrow\\
t_f &= \left(\frac{3v_0}{\dfrac{g}{3t_f^2}}\right)^{1/3} = \left(\frac{3v_0}{1}\frac{3t_f^2}{g}\right)^{1/3}\\
t_f^3 &= \left(\left(\frac{9v_0t_f^2}{g}\right)^{1/3}\right)^3\\
t_f^3\frac{1}{t_f^2} &= \frac{9v_0\cancel{t_f^2}}{g}\frac{1}{\cancel{t_f^2}}\\
&\Downarrow\\
t_f &= \frac{9v_0}{g}
\end{align*}
Finally we have an expression for $t_f$ that we can plug into equation \ref{xEqn}! Now we 
find the distance it takes to come to a stop, $x(t=t_f) - x_0$. But first we need to
replace $c$ into $x(t)$.
\begin{align*}
x(t) &= -\frac{c}{12}t^4 + v_0t + x_0\\
&\Downarrow\\
x(t) &= -\frac{\frac{g}{3t_f^2}}{12}t^4 + v_0t + x_0\\
x(t) &= -\frac{g}{36t_f^2}t^4 + v_0t + x_0
\end{align*}
Now we can find the stopping distance $x(t=t_f) - x_0$ by
\begin{align*}
x(t=t_f) - x_0 &= -\frac{g}{36t_f^2}(t_f)^4 + v_0(t_f) + \cancel{x_0} - \cancel{x_0}\\
&\Downarrow\\
x(t=t_f) - x_0 &= -\frac{g}{36}t_f^2 + v_0t_f\\
&\Downarrow\\
x(t=t_f) - x_0 &= -\frac{g}{36}\left(\frac{9v_0}{g}\right)^2 + v_0\frac{9v_0}{g}\\
&= -\frac{g}{36}\frac{81v_0^2}{g^2} + \frac{9v_0^2}{g}\\
&= -\frac{81v_0^2}{36g}  + \frac{9v_0^2}{g}\\
&= \frac{v_0^2}{g}\left(9-\frac{81}{36}\right)\\
&= \frac{27}{4}\frac{v_0^2}{g}
\end{align*}

We see that in the end result the distance the train travels to a stop does not depend on 
$x_0$! This is because we are interested in the distance traveled relative to $x_0$. So we
can choose any value for $x_0$ and it would not effect the final solution. 
Note that we can check to make sure that the units work out. We know that the units of $x$ 
are in meters. And the units of $v_0^2/g$ are $\frac{(m^2/s^2)}{(m/s^2)} = m$ so our units
work! Now we can plug in our values where $v_0 = 89\unit{m/s}$ and $g=9.8\unit{m/s^2}$ we 
get
\begin{align*}
x(t=t_f) - x_0 &= \frac{27}{4}\frac{v_0^2}{g}\\
&\Downarrow\\
x(t=t_f) - x_0 &= \frac{27}{4}\frac{(89)^2}{9.8} \approx 5500\unit{m}
\end{align*}

Now for extra practice try solving for the more general case where $a(t) = ct^{m}$ and 
$a_{max} = g/k$ and see if you can recreate the general solution
$$x(t_f) - x_0 = \frac{v_0^2(m+1)^2k}{g(m+2)}$$
\end{document}

