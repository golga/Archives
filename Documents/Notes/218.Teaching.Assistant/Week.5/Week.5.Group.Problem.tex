\documentclass[11pt]{article}

\usepackage{latexsym}
\usepackage{amssymb}
\usepackage{amsthm}
\usepackage{enumerate}
\usepackage{amsmath}
\usepackage{cancel}

\setlength{\evensidemargin}{.25in}
\setlength{\oddsidemargin}{-.25in}
\setlength{\topmargin}{-.75in}
\setlength{\textwidth}{6.5in}
\setlength{\textheight}{9.5in}
\newcommand{\due}{September 12th, 2015}
\newcommand{\HWnum}{1}
\input{"C:/Users/joeor_000/Documents/Coursework/Define.tex"}

\begin{document}
\bf{PHY218} \ \ \ \ \ \ \bf{RECITATION WEEK 4} \ \ \ \ \ \ \ \ \ \ \ \ \ \ \bf{YOUR NAME:}
\\

\ \ \ \ \ \ \ \ \ \ \ \ \ \ \ \ \ \ \bf{GROUP QUIZ} \ \ \ \ \ \ \ \ \ \ \ \ \ \ \ \ \ \ \ \ \bf{GROUP NUMBER:}

\noindent\makebox[\linewidth]{\rule{\paperwidth}{0.4pt}}

You tie a rope to a pole placed at the center of an ice rink and skate around it in a 
horizontal circular motion at a constant speed. You tested the rope's strength and found that
it can hold a maximum mass, $M$, before breaking.

\begin{enumerate}[(a)]
\item You use a rope of length, $R$, from the center pole and, you are a mass of $m$, find 
an expression for the maximum velocity you can travel, $v_{max}$, without breaking the rope 
in terms of $M$, $m$, $R$, and $g$. What assumptions about the rope and ice did you make to 
solve this problem? Be sure to include a diagram of the system with all the relevant
quantities and their directions.


\item In the previous part we assumed that we were traveling on frictionless ice. Now assume
that the ice has a coefficient of kinetic friction, $\mu_k$. What is the expression for 
$v_{max}$ now (in terms of $M$, $m$, $R$, $g$, and $\mu_k$)? 

\item Why do we use $\mu_k$ rather than $\mu_s$, the coefficient of static friction?

\end{enumerate}




\end{document}
