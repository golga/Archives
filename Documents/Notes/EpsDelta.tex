\documentclass[11pt]{article}

\usepackage{latexsym}
\usepackage{amssymb}
\usepackage{amsthm}
\usepackage{enumerate}
\usepackage{amsmath}
\usepackage{cancel}

\setlength{\evensidemargin}{.25in}
\setlength{\oddsidemargin}{-.25in}
\setlength{\topmargin}{-.75in}
\setlength{\textwidth}{6.5in}
\setlength{\textheight}{9.5in}
\newcommand{\due}{September 12th, 2012}
\newcommand{\HWnum}{1}
\input{/home/joey/Documents/Class/Define.tex}

\begin{document}
Given the function
$$g(x,y) = \left\{\begin{array}{cc}
           \frac{\sin(x^2+y^2)}{\sqrt{x^2+y^2}}    &\{x,y\}\ne\{0,0\}\\ 
           0                                       &\{x,y\}=\{0,0\}
           \end{array}\right.$$
we can test if $g(x,y)$ is continuous by using an epsilon delta proof. 

Note that the $\epsilon$, $\delta$ definition of the limit states that for a function $f$ the statement
\begin{equation}
\lim_{\{x,y\}\rightarrow \{a,b\}}f(x,y) = L
\label{lim}
\end{equation}
implies that for all $\epsilon>0$ there exists a $\delta>0$ such that for all $x$ within the condition
$$\sqrt{(x-a)^2+(y-b)^2} < \delta$$
the following condition 
$$|f(x,y) - L|<\epsilon$$
also holds true. So we can apply this fact to the definition of continuity
\begin{equation}
\lim_{\{x,y\}\rightarrow \{a,b\}}f(x,y) = f(a,b)
\label{conti}
\end{equation}
by noting that by combining equations \ref{lim} and \ref{conti} we can say $f(x,y)$ is continuous if $L=f(a,b)$. 

So to preform an $\epsilon$-$\delta$ proof of continuity we need to show that the limit of $f(x,y)$ exists and equals the $f(a,b)$. To prove this we must show that given the condition 
$$\sqrt{(x-a)^2+(y-b)^2} < \delta$$
we can find some $\epsilon>0$ that also gives the condition
$$|f(x,y) - L|<\epsilon$$

So we can apply this to $g(x,y)$ at the point $(0,0)$ by first noting that we are given the value $g(0,0) = 0$. This implies that if we can show that
$$\lim_{\{x,y\}\rightarrow \{0,0\}}g(x,y) \stackrel{?}{=} 0$$
then we can say $g(x,y)$ is continuous at the point $(0,0)$. This means that we need to show that there exists some $\epsilon > 0$ that constrains $g(x,y)$ by
$$|g(x,y) - 0|<\epsilon$$
given that 
$$\sqrt{(x-0)^2+(y-0)^2} < \delta$$
So we start with 
$$\left|g(x,y)-0\right| = \left|\frac{\sin(x^2+y^2)}{\sqrt{x^2+y^2}}\right|$$
Note that if we define a new variable $r$ such that $r^2 = x^2+y^2$ we can see that we get
$$\left|\frac{\sin(x^2+y^2)}{\sqrt{x^2+y^2}}\right| = \left|\frac{\sin(r^2)}{r}\right|$$
Note that in the single variable case we can apply \emph{L'H\^{o}pital's rule} to find the limit
$$\lim_{r\rightarrow0}\frac{\sin(r^2)}{r} = \lim_{r\rightarrow0}\frac{2r\cos(r^2)}{1} = 0$$
Due to the fact that the limit exists (we proved it using \emph{L'H\^{o}pital's}) we can can say that there exists some $\delta>0$ and some $\epsilon>0$ such that
\begin{align*}
|r| &< \delta\\
\left|\frac{\sin(r^2)}{r} - 0\right| &< \epsilon
\end{align*}
Now by going back to the variables $x$ and $y$ we see that we have
$$|\sqrt{x^2+y^2}| < \delta$$
note given this condition it follows that
$$\left|\frac{\sin(x^2+y^2)}{\sqrt{x^2+y^2}} - 0\right| < \epsilon$$
This implies that 
$$\lim_{\{x,y\}\rightarrow \{0,0\}}g(x,y) = 0 = f(0,0)$$
Therefore $g(x,y)$ is continuous at the point $(0,0)$.





\end{document}

