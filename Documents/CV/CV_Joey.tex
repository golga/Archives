\documentclass[11pt,letter]{moderncv}

\moderncvtheme[blue]{classic}

\usepackage[utf8]{inputenc}

\usepackage[bottom = 0.75in]{geometry}

% personal data
\firstname{Joe}
\familyname{Becker}
\address{1501 Harvey Road Apt. \#615}{College Station, TX 77840}
\phone{(970)402-3968}
\email{jbecker at physics.tamu.edu}

\newcommand{\up}[1]{\ensuremath{^\textrm{\scriptsize#1}}}

% the ConTeXt symbol
\def\ConTeXt{%
  C%
  \kern-.0333emo%
  \kern-.0333emn%
  \kern-.0667em\TeX%
  \kern-.0333emt}
\definecolor{web}{rgb}{0.2,0.2,0.2}
%\definecolor{web}{rgb}{0.5,0.5,0.5}
%----------------------------------------------------------------------------------
%            content
%----------------------------------------------------------------------------------
\begin{document}
\maketitle
\section{\textbf{Education}}
\cventry{2015--Present}{\textbf{Doctor of Philosophy}}{Texas A\&M University}{College Station, TX}{}{- Physics}
\cventry{2005--2012}{\textbf{Bachelor of Arts}}{University of Colorado}{Boulder, CO}{}{- Physics, over all GPA: 3.0/4.0.}
\cventry{2005--2012}{\textbf{Bachelor of Arts}}{University of Colorado}{Boulder, CO}{}{- Mathematics, over all GPA: 3.1/4.0.}
\cventry{2001--2005}{\textbf{International Baccalaureate Diploma}}{Poudre High School}{Fort Collins, CO}{}{}{}

\section{\textbf{Academic Background}}
\cvitem{Physics}{Advanced Physics/Optics Lab, Junior Level Electronics Lab, Quantum Mechanics, Electricity and Magnetism, Classical Mechanics, Thermodynamics, Error Analysis, Solid State Physics, General Relativity}
\cvitem{Mathematics}{Calculus, Mathematical Analysis, ODE \& PDE,
Complex Analysis, Fourier Analysis, Linear Algebra, Probability Theory, Mathematical Statistics}
\cvitem{Computer Science}{Data Structures, Algorithms}

\section{\textbf{Research Experience}}
\cventry{2006--2008}{\textbf{Research Assistant}}{University of Colorado at Boulder: High Energy Physics BaBar Group}{Professors James G. Smith \& William T. Ford}{}
{- Preformed data analysis for the BaBar collaboration. \newline
 - Measured quasi-twobody decays $B^0\rightarrow a_0(1450)^-\pi^+$, $B^0\rightarrow a_0(1450)^-K^+$, and $B^0\rightarrow \eta\rho^0$}

\cventry{2011}{\textbf{Summer Internship}}{Tech-X Corporation}{Peter Stoltz Ph.D}{}
{- Conducted a verification study on Nautilus, the fluid plasma modeling software. \newline
- Data analysis using python specifically in the NumPy, SciPy, MatPlotLib environment.}

\cventry{2012--2013}{\textbf{Research Assistant}}{Liquid Crystal Materials Research Center}{Professors Noel Clark, Matthew Glaser, \& Joseph Maclennan}{}
{- Designed and conducted scientific measurements on free-suspended liquid crystal films. \newline
- Data analysis on experimental data using Python, Mathematica, MatLab, \& Origin 9}

%\cventry{2013--2014}{\textbf{AM Sous Chef}}{Under the Sun Eatery \& Taphouse}{Tim McMurray}{}
%{- Food preparation, Kitchen and Inventory management}

\cventry{2014--2015}{\textbf{Research Assistant}}{National Institute of Standards and Technology}{Scott B. Papp \& Scott A. Diddams}{}
{- Researched low noise stimulated Brillouin scattering lasing using silica microrod resonators. \newline
- Whispering gallery mode micro-resonator construction and analysis. \newline
- Created poster and talk for presentation at the International Frequency Control Symposium 2015.}

\section{\textbf{Publications}}
\cventry{2007}{The BABAR Collaboration, B. Aubert, et al}{"Search for Neutral B-Meson Decays to a0pi, a0K, etarho0, and etaf0"}{Phys. Rev D \textbf{75}, 111102 (2007)}{}{}{}{}

\cventry{2015}{J. Becker, W. Loh, F. Baynes, D. Cole, F. Quinlan, H. Lee, K. Vahala, S. Papp, S. Diddams}{"Toward Chip Integrated Ultra-Low-Noise Lasing Using a Microrod Resonator"}{International Frequency Control Symposium 2015}{}{}{}{} 

\cventry{2015}{W. Loh, J. Becker, F. Baynes, D. Cole, F. Quinlan, H. Lee, K. Vahala, S. Papp, S. Diddams}{"Low-Noise Stimulated  Brillouin Lasing in a Microrod Resonator"}{Conference on Lasers and Electro-Optics 2015}{}{}{}{} 

\section{\textbf{Relevent skills}}
\cvcomputer{OS}{Linux/Unix, Windows, DOS} {Programming}{C/C++, Python, Perl, IDL}
\cvcomputer{Scientific}{Matlab, Maple, Mathematica, Matplotlib, LabView, Origin 9} {Typography}{\LaTeX, Microsoft Office, Inkscape}
\cvcomputer{Miscellaneous}{Precision Machining}{}{}

\end{document}
