\section{Overview}
The metastable singlet state of the negatively charged nitrogen—vacancy (NV$^-$) defect in diamond underpins a huge swath of measurement techniques that utilize the nitrogen—vacancy diamond (NVD).  It is therefore of critical importance to resolve open problems regarding the singlet state energy level structure and to elucidate the effects of temperature and the wavelength of infrared (IR) excitation on the optically detected magnetic resonance (ODMR) contrast arising from fluorescence quenching transitions through the singlet state.  As a part of the proposed research, we plan to resolve the open questions regarding the singlet state energy level structure by demonstrating a novel singlet energy level diagnostic technique using a photoionization regime.  We also propose to differentiate between the effects of temperature and IR excitation on the ODMR contrast by plotting the dependence of ODMR contrast on temperature and wavelength of IR excitation.  We also propose to leverage this information to create methods for enhanced sensitivity deep tissue magnetic field and temperature diagnostics, and super-resolution imaging regimes.  The proposed research program includes five work packages.  In the first work package we plan to demonstrate photoionization from the singlet state as a tool for measuring the separation of the singlet state from the conduction band edge.  The second work package will entail a demonstration of enhanced contrast ODMR via a continuous wave singlet state photoionization regime.  The third work package will utilize the results of the first two work packages to enhance the sensitivity of our fiber-optic NVD probes for deep tissue magnetic field and temperature diagnostics.  Our fourth work package will lay the groundwork for temperature sensitive background free super-resolution imaging by determining the dependence of the ODMR contrast on the NVD temperature and wavelength of IR excitation.  Finally, the fifth work package will demonstrate our temporally modulated temperature sensitive stimulated emission depletion (STED) regime capable of super-resolution thermally sensitive imaging.

\section{Intellectual Merit}
A better understanding of the singlet state energy level structure affords the possibility of quantum sensors with radically enhanced sensitivity to magnetic fields and temperature distributions.  Enhanced knowledge of the dependence of ODMR contrast on temperature will give rise to the prospect of temperature sensitive super-resolution imaging immune to background autofluorescence.  The proposed measurement techniques will resolve important open questions about the NV$^-$ singlet state energy level structure and illuminate the previously unexplained dependence of the ODMR contrast on temperature and the wavelength of infrared excitation.