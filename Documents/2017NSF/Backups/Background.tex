\section{Rationale and Objectives}
Recent studies of the energy level structure of the negatively charged nitrogen—vacancy center (NV$^-$) in diamond have 
left much controversy concerning the energy level distribution of the short-lived $^1A$ singlet state and the metastable $^1$E singlet state (henceforth referred to as the singlet state) within the band-gap 
of the diamond lattice.  The location of these energy states within the band-gap is of fundamental importance in the 
drive to understand the spin-selective shelving process that ultimately underlies most applications of the nitrogen—
vacancy diamond (NVD).  A thorough understanding of the energy level distribution of the NV$^-$ singlet state manifold may 
allow for novel measurement regimes employing spin-selective ionization of the NV$^-$ from the singlet state, affording 
the prospect of enhanced optically detected magnetic resonance (ODMR) contrast for higher sensitivity measurements 
of magnetic fields and temperature distributions.  Another open question is the degree to which the ODMR contrast 
depends on the wavelength of radiation targeted at the infrared (IR) transitions within the NV$^-$ singlet state 
manifold versus its dependence on the temperature of the NVD.   There are many open questions regarding the phonon 
sideband structure of the NV$^-$ singlet state that which could be explored using the dependence of ODMR contrast on 
the wavelength of radiation targeted at the IR transitions within the NV$^-$ singlet state manifold.  Knowledge of the 
dependence of ODMR contrast on NVD temperature could facilitate temperature sensitive lock-in detection assisted stimulated emission depletion (STED) imaging using NVD nanoparticles by employing a pulsed microwave source 
targeted at the zero-field ground state sublevel splitting for modulating NVD fluorescence, and using the dependence of the ODMR contrast on temperature to generate super-resolution images with temperature sensitivity.

As a part of the proposed research, we plan to investigate the separation of the NV$^-$ singlet state from the 
conduction band edge by employing a wavelength-dependent photoionization regime involving pulsed microwave and 
wavelength-tunable laser excitation.  The singlet state of an NV$^-$ will be populated through a spin-selective 
intersystem crossing (ISC) with sequential microwave and laser pulses, whereupon a pulse of wavelength-tunable laser 
radiation will be applied to the NV  until it is ionized into a neutrally charged NV center (NV$^0$).  This spin-
selective photoionization ties the fluorescence excited from the subsequent NV$^0$ to the population in the NV$^-$ singlet 
state as a function of laser wavelength, thereby allowing for determination of the distance of the singlet state 
from the edge of the conduction band by identifying the peak in NV$^0$ fluorescence intensity that corresponds to 
single photon ionization of the NV$^-$ from the singlet state.  In a subsequent section of the proposed research, we 
will utilize the knowledge gained from determining the singlet ionization energy to create a novel technique for 
enhanced contrast ODMR imaging of magnetic fields and temperature distributions.

We also propose to explore the dependence of the ODMR contrast on both temperature and IR wavelength.  By pumping 
the NV$^-$ into the singlet state with laser radiation and resonant microwave excitation, we can test the dependence of 
the ODMR contrast on temperature by attaching a thermocouple to the NVD and heating the diamond with a heat source 
to generate a plot of ODMR contrast as a function of temperature.  We propose an additional experiment that will 
explore the dependence of the ODMR contrast on the wavelength of IR radiation applied to the NV$^-$ while it is in the 
singlet state by fixing the power from an IR source, varying the applied IR wavelength, and generating a plot of 
ODMR contrast as a function of IR wavelength.  By comparing the two plots in these experiments we hope to 
differentiate the effects of IR radiation and temperature on ODMR contrast.  We plan to leverage this information to 
create a novel temperature sensitive STED regime involving fluorescence amplitude modulation with pulsed microwave 
radiation, lock-in detection to suppress the background autofluorescence that plagues many STED techniques, and 
analysis of the thermally induced change in ODMR contrast as a simultaneous measure of the local temperature 
environment.

The proposed research program includes five work packages.  In the first work package, we propose to determine 
the separation of the NV singlet state from the conduction band by using photoionization as a function of applied 
laser wavelengths.  Research within the second work package will be focused on demonstrating enhanced contrast 
ODMR.  Research within the third 
work package will include incorporation of the enhanced contrast ODMR imaging technique to improve the 
sensitivity of existing fiber-optic NVD probes that enable deep tissue diagnostics of weak magnetic fields and 
temperature distributions of a biological origin.  The fourth work package will develop separate techniques for 
measuring the dependence of the ODMR contrast on temperature and for measuring the dependence of the ODMR 
contrast on the wavelength of IR radiation applied when the NV$^-$ is in the singlet state.  The fifth work package 
will leverage the information gained from the fourth work package to develop a temporally modulated STED regime 
involving modulation of NV$^-$ fluorescence amplitude with a pulsed microwave source and lock-in detection for background insensitive super-
resolution microscopy with temperature sensitivity as a function of ODMR contrast.