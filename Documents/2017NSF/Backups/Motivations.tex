\section{State of the Art and Motivation}
A NV color center in diamond is a unique solid-state quantum system \citep{Gaebel2006,Dutt2007,Aharonovich2011}, where the electron
spin can be manipulated, polarized, and read out at room temperature using electromagnetic
fields. The electron spin of NV centers in diamond display an extraordinarily long-lived
coherence even at room temperatures, \cite{Gaebel2006,Dutt2007,Aharonovich2011,Childress2006} enabling coherence-control- enhanced quantum data
processing \cite{Dutt2007,Aharonovich2011} and offering much promise as solid-state qubits \cite{Dutt2007,Aharonovich2011,Childress2006,Nizovtsev2005}, single-photon sources \cite{Beveratos2002}
, efficient contrast agents for super-resolution microscopy \cite{Gruber1997}, and photostable, nonbleaching
markers for bioimaging \cite{Gruber1997,McGuinness2011} including high spatial and temporal resolution imaging of neural
activity \cite{Hall2012}. The sensitivity of the electron spin in an NV center to an external magnetic field has
been shown to enable a new approach in magnetometry, allowing weak magnetic fields to be
detected and imaged with an unprecedented spatial resolution and a remarkable sensitivity \cite{LeSage2013,Taylor2008,Maze2008,Balasubramanian2008}
The temperature sensitivity of magnetic-resonance spectra of NV$^-$ \cite{Acosta2010}
has been shown to enable a new modality of optical thermometry, allowing temperature
measurements with a millikelvin accuracy and an unprecedented, nanometer-scale spatial
resolution \cite{Kucsko2013}, thus offering a unique tool for a thermometry of living cells.

The sensitivity of these techniques depends on the value of the ODMR contrast, which
has historically been capped at 30\%. Recently developed techniques involving spin-to- charge
conversion via photoionization and single-shot charge state readout have improved ODMR
contrast dramatically \cite{Shields2015}. However, the charge state readout method in this technique relies on
low excitation power and photon counting, and therefore suffers from shot-noise limited
sensitivity. This limitation can be overcome by a continuous wave ionization regime targeted at
the singlet state ionization wavelength and reading out NV$^0$ fluorescence. Recent work has left
some uncertainty regarding the separation of the singlet state from the conduction band edge \cite{Toyli2012,Goldman2015,Goldman2015a}. 
To redress this uncertainty, a technique for determining the singlet ionization wavelength
must be developed, and singlet ionization must be demonstrated experimentally.

While it is a well-known fact that the total fluorescence amplitude decreases as a function of NVD temperature \cite{Lai2013}, the dependence of the ODMR 
contrast on NVD temperature versus its dependence on the wavelength of applied IR excitation is a source of uncertainty \cite{Blakley2016}.  Resolving this 
uncertainty is of key importance to developing novel NVD based STED regimes capable excluding background autofluorescence while simultaneously 
measuring the temperature of the environment around the NVD.

The STED technique has been a mainstay of super-resolution microscopy for more than a decade
\cite{Hell2003,Hell2009,Eggeling2009}.  STED techniques using fluorescent nitrogen 
vacancy particles have yielded point-spread functions of only 5.8 nm \cite{Rittweger2009,Han2009}, 
affording the ability to resolve individual NV centers.  One 
challenge of STED microscopy has been separating the signal from a bright fluorescence background.  This has be achieved with NVD particles by 
modulating the fluorescence amplitude with microwave radiation or a magnetic field and employing lock-in detection to separate the signal from the 
background \cite{Chapman2013,Sarkar2014}.  In our earlier work, a technique for temporal modulation of the STED signal was developed by modulating the amplitude of the 
near infrared (NIR) STED field and using lock-in detection to improve signal-to-noise ratio \cite{Doronina-Amitonova2015}.  Using an NVD based STED regime offers the 
attractive possibility of achieving super-resolution microscopy with temperature sensitivity for a unique look at biological systems.